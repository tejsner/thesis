\chapter*{Resum\'e}
Forståelsen af høj-temperatur superledning i kupraterne er stadig et uløst spørgsmål i faststoffysikken. Ved at huldope det Mott-insulatorende materiale La$_2$CuO$_4$ med forskellige kemiske arter, skabes en superleder med en overgangstemperatur $T_\text{c}$ der afhænger kraftigt af typen og mængden af denne kemiske `dopant'. Selvom mængden af huldoping typisk dikterer fasediagrammet, er der situationer hvor superledning kan understrykkes eller forbedres på baggrund af strukturelle modifikationer forudsaget af den kemiske dopant. I denne afhandling undersøges disse strukturelle modifikationer, og især den relaterede dynamik, for bedre at forstå en eventuel relation til superledning.

En særligt interessant prøve i denne sammenhæng er La$_{2-x}$Sr$_x$CuO$_{4+\delta}$ (LSCO+O), hvor doping kan udføres med to forskellige kemiske arter, Sr og O. Doping med Sr er `quenched', hvilket betyder at Sr har en fikseret, tilfældig distribution. På den anden side, udføres doping med O ved brug af elektrokemiske metoder after krystallen er groet. Oxygen doping er derfor `annealet' i den forstand at O sidder på interstitielle sites og er mobile ved stuetemperatur. Derudover skaber doping med O en superleder med lidt bedre $T_\text{c}$ der er udstyret med en række unikke fænomener som elektronisk faseseperation og komplekse superstrukturer. I denne afhandling undersøges diverse aspeketer af fonon-dynamik i LSCO+O gennem en komination af teoretiske og eksperimentelle metoder. Density Functional Theory og molekyledynamik bruges til at estimere fonon båndstrukuter og density-of-states. Inelastisk neutron spredning bruges til at undersøge fonon dynamik eksperimentelt og de to metoder sammenlignes.

Selvom Density Functional Theory ikke kan beskrive den elektroniske struktur af kuprater, finder vi en fremragende overensstemmelse mellem simulering og eksperiment i kontekst af fonon-dynamik. En analyse af denne overensstemmelse afslører unik dynamik af octahedrale tilt-mønstre som konsekvens af interstitiel oxygen. Vi formoder at disse tilt-mønste kan assistere til at lindre den kendte frustration mellem magnetisme og superledning i LSCO+O. Afslutningsvis undersøger vi en såkaldt fonon-anomali relateret til en Cu-O vibration i to forskellige LSCO+O prøver ved hjælp af neutronspredning. Denne fonon-anomaly mistankes for at være relateret til ladningsfluktationer som igen mistænkes for at være vigtigt for superledning. Selvom denne anomali tidligere er observeret i Sr-dopet La$_{2-x}$Sr$_x$CuO$_4$, udelukker dette resultat en fonon-anomali forudsaget af strukturelle mekanismer relatret til den specifikke dopant.
