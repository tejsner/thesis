\newcommand{\jp}{j^\prime}
\newcommand{\jpp}{j^{\prime\prime}}
\newcommand{\lp}{l^\prime}
\newcommand{\lpp}{l^{\prime\prime}}
\newcommand{\fc}{\bm{\Phi}\genfrac{(}{)}{0pt}{}{j \jp}{l \lp}}
\newcommand{\fczero}{\bm{\Phi}\genfrac{(}{)}{0pt}{}{j \jp}{0 \lp}}
\newcommand{\fcb}{\bm{\Theta}\genfrac{(}{)}{0pt}{}{j \jp}{l \lp}}
\newcommand{\fcbpp}{\bm{\Theta}\genfrac{(}{)}{0pt}{}{j \jpp}{l \lpp}}
\newcommand{\fcbf}{-\bm{\Theta}\genfrac{(}{)}{0pt}{}{j \jp}{l \lp} + \delta_{j,\jp} \delta_{l,\lp} \sum_{\jpp, \lpp}  \bm{\Theta}\genfrac{(}{)}{0pt}{}{j \jpp}{l \lpp} }
\newcommand*\tageq{\refstepcounter{equation}\tag{\theequation}}

\chapter{Methods}\label{ch:method}

\begin{framed}
	\begin{itemize}
		\item Neutron scattering (first, since we can then get simulation cross-sections)
		\item Specific neutron scattering methods (TAS (IN8, ThALES), PDF (D4), DOS (IN4)). Give the specific equations such as phonon cross-sections here (so we can refer to them later.)
		\item Density Functional Theory (Basic principles, KS equations, HK Theorem, weaknesses of one-electron theory)
		\item Jacob's Ladder (and how we circumvent it with DFT+U)
		\item Forces in DFT
		\item Phonons with DFT
		\item Molecular dynamics
		\item Simulation and Experiment summary (what can we simulate/measure and how can we compare)
	\end{itemize}	
\end{framed}

\section{Neutron Scattering}

\subsection{General Theory}

\subsection{Specific Scattering Methods}

\section{Density Functional Theory}\label{sec:dft}

A single DFT self-consistent cycle `only' spits out the total energy, the charge density and the wavefunction parameters. From this output we can obtain additional information such as the electronic band structure, electronic density of states, forces on atoms and even information about chemical bonds \cite{Silvi1994}. It becomes the job of the scientist to ask the right questions and set up a series of simulations to fit his or her purpose.

\subsection{The many-body wavefunction}
In electronic structure calculations we are ultimately interested in the many-body wavefunction

\[ \Psi(\bm{r}_1,\bm{r}_2,\dots, \bm{r}_n; \bm{R}_1, \bm{R}_2, \dots , \bm{R}_N) \]

\noindent where $\bm{r}_i$ are electron coordinates and $\bm{R}_i$ are nuclear coordinates. Imagine that we want to calculate this object for a small molecule such as Benzene (C$_6$H$_6$) containing 12 nuclei and 42 electrons. This wavefunction exists in $42\cdot3-6 = 156$ dimensional cartesian space! If we want to store this object on a computer with a modest precision of 10 grid points per coordinate, it would require $10^{156}$ complex numbers or $64 \cdot 10^{156}$ bits (assuming single-precision floating points numbers of 32 bits). Lloyd2000 estimated the total number of bits available for computation in the observable universe to be $10^{90}$ (Lloyd2000). Even with the entire universe at our disposal this object is completely unmanageable. This exercise also emphasizes the potential of Quantum Computers. For this reason, we fix the ionic positions and assume that the many-body wavefunction can be written as a product of single-electron wavefunctions (orbitals):

\[ \Psi(\bm{r}_1,\bm{r}_2,\dots, \bm{r}_n; \bm{R}_1, \bm{R}_2, \dots , \bm{R}_N) \longrightarrow \phi(\bm{r}_1)\phi(\bm{r}_2)\dots\phi(\bm{r}_n) \]

\subsection{The Kohn-Sham equations}

\subsection{Jacob's Ladder}

\subsection{DFT+U}

\subsection{The Hellman-Feynman theorem and molecular forces}

\section{Phonon Calculations}
In most textbooks (e.g. Kittel \cite{Kittel2005}), phonon calculations are exemplified by simple models in one dimension consisting of only one or two inequivalent atoms. While these models are useful for providing basic results of lattice dynamical models, the extension to realistic models requires some level of abstraction in order to be useful. In particular, it is essential to cast the problem in terms of linear algebra. In this section, I will start from the (somewhat abstract) formalism used in practice and work backwards towards a physical understanding. While software such as PHONON \cite{Parlinski1997} and Phonopy \cite{Togo2015} can be used without prior knowledge of the formalism, it is always useful to have some insights about our frequently used 'black boxes`. In order to calculate the phonon spectrum for a given system in the harmonic approximation, we require the following objects:

\begin{enumerate}
	\item Primitive unit cell and fractional atomic coordinates
	\item Symmetry operations
	\item The mass of each atomic species
	\item The force constants
\end{enumerate}

Items 1-3 are familiar to most condensed matter physicists and can usually be found in various databases. The force constants, on the other hand, contains information about interatomic forces and is not directly obtainable from experiment. For this reason, phonon calculations requires some modelling either through semi-empirical or ab-initio methods. In the following I will attempt to explain what the force constants represents and how we use them to get phonon band structures.

\subsection{Theory}

We start completely generally in one dimension with an arbitrary number of unit cells containing an arbitrary number of atomic species at equilibrium. Displacements from equilibrium positions are denoted $u(jl)$, where $l$ is the unit cell index and $j \in \{1,\dots,n\}$ is the atomic index. If we consider the displacements $u$ to be small, the total energy of our system can be expressed as a Taylor series

\[ E^\text{tot} = E_0 + \sum_{l}\sum_{j} \left. \frac{\partial E}{\partial u(jl)} \right\rvert_{r_{lj}}  + \frac{1}{2} \sum_{l,\lp} \sum_{j,\jp} u(jl) \left. \frac{\partial ^2 E}{\partial u(jl) \partial u(\jp \lp)} \right\rvert_{r_{lj}, r_{\lp \jp}} u(\jp \lp) + \dots \, , \]

\noindent where $r_{lj}$ is the equilibrium position of atom $j$ in unit cell $l$. The main approximation in phonon calculations is the so-called \emph{harmonic approximation} which ignores terms with power greater than 2 in the series. Higher-order contributions are denoted \emph{anharmonic} terms and can become important at higher temperatures (phase transitions, thermal conductivity, thermal expansion). The fact that our system is in equilibrium can be stated succinctly as

\[ \frac{\partial E}{\partial u(jl)} = 0 \, , \]

\noindent for all values of $j$ and $l$. Physically these assumptions together correspond to atoms being at rest in a parabolic (harmonic) potential. Since we are interested in  dynamics, it is convenient to consider the \emph{harmonic energy} $E$ of the system

\begin{equation}
E = E^\text{tot} - E_0 = \frac{1}{2} \sum_{l,\lp} \sum_{j,\jp} u(jl) \left. \frac{\partial ^2 E}{\partial u(jl) \partial u(\jp \lp)} \right\rvert_{r_{lj}, r_{\lp \jp}} u(\jp \lp) \label{eq:eharm}
\end{equation}

\noindent If we set $j=\jp$ and $l=\lp$, we see that the harmonic energy of a single atom has the familiar form of a harmonic oscillator $E=\frac{1}{2}Ku^2$, where $K$ is the spring constant. We define

\[ \left. \frac{\partial ^2 E}{\partial u(jl) \partial u(\jp \lp)} \right\rvert_{r_{lj}, r_{\lp \jp}} = \fc = \fcbf \]

\noindent where $\bm{\Phi}$ is the called the \emph{force constant} with respect to total energy and $\bm{\Theta}$ is the force constant with respect to bond energy. We can now write the harmonic energy as

\begin{align*}
E &= \frac{1}{2} \sum_{l,\lp} \sum_{j,\jp} u(jl) \fc u(\jp \lp) \tageq\label{eq:total_energy} \\
&= \frac{1}{2} \sum_{l,\lp} \sum_{j,\jp} u(jl) \left( \fcbf \right) u(\jp \lp) \\
&= \frac{1}{2} \sum_{l,\lp} \sum_{j,\jp} \left( - \fcb u(jl)u(\jp,\lp) + \fcb u(jl)^2 \right) \\
&= \frac{1}{2} \sum_{l,\lp} \sum_{j,\jp} \left( - \fcb u(jl)u(\jp,\lp) +\frac{1}{2} \fcb \left( u(jl)^2 + u(\jp, \lp)^2 \right) \right) \\
&= \frac{1}{4} \sum_{l,\lp} \sum_{j,\jp} \fcb \left[ -2u(jl)u(\jp,\lp) + u(jl)^2 + u(\jp,\lp)^2 \right] \\
&= \frac{1}{4} \sum_{l,\lp} \sum_{j,\jp} \fcb \left[ u(jl) - u(\jp,\lp) \right]^2
\end{align*}

\begin{figure}
	\centering
	\includegraphics[width=0.9\textwidth]{fig/temp/diatomic.png}
	\caption[diatomic chain]{Diatomic chain. \todo[inline]{Make new figure. Use $l$ as unit cell index for consistency with the notation.}}
	\label{fig:diatomic}
\end{figure}

\noindent and it becomes evident that the harmonic energy can be described with respect to atoms or bonds in mathematically equivalent ways. Since the bond-centered description does not include individual atomic displacements, it is necessary to add a self-term to $\bm{\Theta}$. As a visual aid to these index-heavy equations, Figure \ref{fig:diatomic} illustrates the one-dimensional diatomic chain, which is often used in introductory texts. If we consider only nearest-neighbour interactions and identical springs, the bond-centered harmonic energy can be written

\begin{align*}
E &= \frac{1}{4} \bm{\Theta} \sum_l 2 \left[ u(1,l) - u(2,l) \right]^2 + \frac{1}{4} \bm{\Theta} \sum_n 2 \left[ u(2,l) - u(1,l+1) \right]^2 \\
&= \frac{1}{2} \bm{\Theta} \sum_l \left[ u(1,l) - u(2,l) \right]^2 + \frac{1}{2} \bm{\Theta} \sum_n \left[ u(2,l) - u(1,l+1) \right]^2
\end{align*}

\noindent where the factor of 2 comes from double-counting. The purpose of this example is to show that the (somewhat abstract) harmonic energy in equation \eqref{eq:eharm} is equivalent to our intuitive understanding of coupled harmonic oscillators. With this in mind, we can return to the matter at hand and write the equation of motion for an atom $j$ in cell $l$ through Newtons second law $F=ma$:

\[ m_j \ddot{u}(jl, t) = - \frac{\partial E}{\partial u(jl)} = - \sum_{\lp} \sum_{\jp} \fc u(\jp \lp, t) \, , \]

\noindent where $m_j$ is the atomic mass of atom $j$\todo{Where does the factor 2 come from when taking the derivative?!?}. Solutions to this equation is given as a sum of travelling harmonic waves with wave vectors $q$ and band indices $\nu \in \{1,\dots , n \}$

\[ u(jl,t) = \sum_{q,\nu} \tilde{u}(j,q,\nu) \exp (iqr(jl)) \exp(-i \omega(q,\nu) t) \, \]

\noindent where $\omega(q,\nu)$ is the frequency, $r(jl)$ is the position of atom $j$ in cell $l$ and  is the frequency and the complex number $\tilde{u}$ is called the \emph{displacement vector}. If we insert these solutions into the equations of motion and just consider one band at one wave vector we obtain

\begin{align*}
m_j \omega(q,\nu)^2 \tilde{u}(j,q,\nu) \exp(ikr(jl)) &= - \sum_{\lp} \sum_{\jp} \fc \tilde{u}(\jp,q,\nu) \exp (ikr(\jp\lp)) \\
m_j \omega(q,\nu)^2 \tilde{u}(j,q,\nu) & = - \sum_{\lp} \sum_{\jp} \fc \tilde{u}(\jp,q,\nu) \exp (ik[r(\jp\lp) - r(jl)]) \\
m_j \omega(q,\nu)^2 \tilde{u}(j,q,\nu) & = - \sum_{\lp} \sum_{\jp} \fczero \tilde{u}(\jp,q,\nu) \exp (ik[r(\jp\lp) - r(j0)]) \tageq\label{eq:motion} \, ,
\end{align*}

\noindent where the last equality is simply a change of origin in order to follow the convention of most software. The full account of phonon frequencies $\omega$ and displacements $\tilde{u}$ can be found as a solutions to equation \eqref{eq:motion}. At a given $q$ and $\nu$, the equations are indexed by $j$ and we will have $n$ equations with $n$ unknowns with respect to $\tilde{u}(j,q,\nu)$, where $n$ is the number of atoms in the unit cell. In fact, equation \eqref{eq:motion} can be written as an eigenvalue equation:

\begin{equation}
\bm{D}(q) \cdot \bm{e}(q,\nu) = \omega(q,\nu)^2 \cdot \bm{e}(q,\nu) \, , \label{eq:dynmat}
\end{equation}

\noindent where 

\[ \bm{e}(q,\nu) = \begin{pmatrix}
\sqrt{m_1}\tilde{u}(1,q,\nu) \\
\sqrt{m_2}\tilde{u}(2,q,\nu) \\
\vdots \\
\sqrt{m_n}\tilde{u}(n,q,\nu)
\end{pmatrix} \]

\noindent and the elements of $\bm{D}(q)$ are given 

\begin{equation}
D(j\jp) = \frac{1}{\sqrt{m_j m_{\jp}}} \sum_{\lp} \fczero \exp (iq[r(\jp\lp) - r(j0)]) \, . \label{eq:dynmat_ij}
\end{equation}

\noindent $\bm{D}(q)$ is known as the dynamical matrix and can be constructed solely from force constants. Furthermore, equation \eqref{eq:dynmat_ij} reveals that the dynamical matrix is Hermitian so the eigenvalues $\omega(q,\nu)^2$ are real and the eigenvectors $\bm{e}(q, \nu)$ are orthonormal. In addition, the eigenvalues and eigenvectors are trivially obtained numerically (e.g. \texttt{numpy.linalg.eigh} in the Python numpy library). In order to get the full dispersion, this diagonalization is performed for each of the $n$ bands $\nu$ at the desired wave vectors in the first Brillouin Zone (FBZ). 

The extension to 3 dimensions is done by treating the Cartesian components separately and considering $\bm{q}$ and $\bm{r}$ as vectors. The eigenvector then becomes a column vector of $3n$ components 

\[ \bm{e}(\bm{q},\nu) = \begin{pmatrix}
\sqrt{m_1}\tilde{u}_x(1,\bm{q},\nu) \\
\sqrt{m_1}\tilde{u}_y(1,\bm{q},\nu) \\
\sqrt{m_1}\tilde{u}_z(1,\bm{q},\nu) \\
\sqrt{m_2}\tilde{u}_x(2,\bm{q},\nu) \\
\vdots \\
\sqrt{m_{n}}\tilde{u}_z(n,\bm{q},\nu)
\end{pmatrix} \, , \]

\noindent the number of bands increase to $3n$ and we get a $3n \times 3n$ dynamical matrix, where each component \eqref{eq:dynmat_ij} is a $3 \times 3$ block of the form

\[
\bm{D}(j\jp) = \begin{pmatrix}
D(j\jp)_{xx} & D(j\jp)_{xy} & D(j\jp)_{xz} \\
D(j\jp)_{yx} & D(j\jp)_{yy} & D(j\jp)_{yz} \\
D(j\jp)_{zx} & D(j\jp)_{zy} & D(j\jp)_{zz}
\end{pmatrix}
\]

\noindent where

\begin{equation}
D(j\jp)_{\alpha\beta} = \frac{1}{\sqrt{m_j m_{\jp}}} \sum_{\lp} \fczero _{\alpha\beta} \exp (i\bm{q}[\bm{r}(\jp\lp) - \bm{r}(j0)]) \label{eq:dynmat_ij2}
\end{equation}

\noindent While the path was somewhat involved, it is useful to take a step back and consider the consequences of our outlined formalism. Everything we need to know about our phonon system can be obtained from the dynamical matrix that, in turn,  is constructed from force constants through equation \eqref{eq:dynmat_ij2}. Finally, all of these objects can be constructed in computationally trivial way from force constants.

\subsection{Practical considerations}\label{sec:phononpractical}
At this point, it is useful to consider how we construct elements of the dynamical matrix in practice. Inspection of equation \eqref{eq:dynmat_ij2} contains a sum over all unit cells $\lp$ and is thus an infinite sum. On the other hand, it is reasonable to assume that the dominant force constants $\bm{\Phi}$ are short range. The compromise is to use a finite supercell such that the second derivatives involved in calculating force constants outside this cell are minimized. While the reasonable size of such a supercell obviously depends on the system and model, quantum contributions to the force constants generally vanish within a distance of roughly \SIrange{10}{15}{\angstrom} \todo{CRYSTAL website reference? Maybe something better}. If the force constants can be obtained analytically from a semi-empirical potential, calculation is computationally simple and we can use large supercells. However, since force constants are usually obtained from DFT, we are limited by computational resources and are usually restricted to supercells with a maximal interatomic distance of roughly \SI{5}{\angstrom} (e.g. a cubic system with $a=\SI{10}{\angstrom}$).

Since the number of force constants needed is at least equal to the size of the dynamical matrix, the number of calculations to perform is at least $3n \times 3n$. Even for a fairly small system such as LCO in the I4/mmm space group (HTT, $n=7$) the number of elements in the dynamical matrix is $(3\cdot 7)^2 = 441$. In the finite displacement method, each force constant is the result of a self-consistent DFT calculation, so the computational effort appears prohibitively expensive at first glance. For this reason we use a numerical fitting of symmetry inequivalent force constants (see section \ref{sec:forceDFT}). In the case of LCO in the I4/mmm space group the number of necessary displacements is reduced to only 7 (6 if we ignore magnetism), making the problem much more manageable.

\subsection{Phonon eigenvectors}
The phonon dispersion is contained within the eigenvalues $\omega (\bm{q},\nu)^2$. We can plot the bands $\nu$ along high-symmetry lines in the FBZ by carefully choosing the values of $\bm{q}$ where the dynamical matrix is diagonalized. Similarly, we can sample the dispersion in a dense $\bm{q}$-mesh in order to evaluate the phonon density of states. In addition, many thermodynamic properties can be calculated by only considering the eigenvalues.

The eigenvectors $\bm{e}(\bm{q}, \nu)$ are more subtle in nature. Each component of  $\bm{e}(\bm{q}, \nu)$ is a complex number that describes the wave amplitude and phase of one atomic species $j$ in one cartesian direction $\alpha$. In addition, the eigenvector is normalized and thus only describes relative atomic motion. In order to visualize the collective displacement due to a phonon mode $\nu$ at $\bm{q}$ we can displace all atoms $j$ in unit cells $l$ by

\begin{equation}
\Delta_{jl} = \frac{A}{\sqrt{m_j}} \text{Re} \left[ \exp (i\phi) \bm{e}_j(\bm{q}, \nu) \exp (i \, 2 \pi \, \bm{q} \cdot \bm{r}(jl) \right] \label{eq:displacements}
\end{equation}

\noindent where $\bm{e}_j(\bm{q}, \nu)$ is the $j$'th component of $\bm{e}(\bm{q}, \nu)$ and $A$ is an arbitrary amplitude. The phase $\phi$ describes periodic motion of atoms. An animation can be produced by varying $\phi$ between 0 and $2 \pi$.

In \texttt{Phonopy} the eigenvectors are always given with respect to the primitive unit cell and the 3 Cartesian components are along the basis vectors of this primitive unit cell. If we want to, for example, to visualize the bond-stretching mode in the HTT phase of LCO at $q=(\frac{1}{4},\frac{1}{4},0)$ in orthorhombic notation, we look at atoms in primitive unit cells with origin (0,0,0), (1,0,1), (2,0,2) and (3,0,3) using the eigenvectors at $q=(0.125,-0.125,0.125)$. For this specific mode, the movement of oxygen with fractional coordinates (0.5,0,0.5) dominates. The movement is exactly along the Cu-O bond and we can plot it in one dimension. Figure \ref{fig:bs-displacements} shows this phonon mode at the zone center, the zone boundary and halfway through the zone (where the giant phonon anomaly is observed).

\begin{figure}
	\centering
	\includegraphics{fig/temp/bs-phonons.pdf}
	\caption[bond-stretching phonons visualized]{Cu-O bond stretching mode in LCO at three different values of $q$ referring to the primitive HTT (I4/mmm) unit cell. Blue markers are oxygen and red markers are copper. The phase is set to $\phi=\frac{3}{4}\pi$ in equation \ref{eq:displacements} in order to get displacement vectors of equal length.}
	\label{fig:bs-displacements}
\end{figure}

\subsection{Obtaining Force Constants from DFT}\label{sec:forceDFT}
Force constants from DFT are found in a surprisingly simple way. In the previous sections we defined the force constant with respect to total energy as

\[ \fc  _{\alpha \beta} = \frac{\partial ^2 E}{\partial u_\alpha (jl) \partial u_\beta (\jp \lp)} = - \frac{\partial F_\beta (\jp \lp)}{\partial u_\alpha (jl)} \]

\noindent where $F_\beta (\jp\lp)$ is the force on atom $(\jp \lp)$ in the direction $\beta$. Notice that everything is now labelled by a Cartesian direction and these directions are treated individually. In practice the Cartesian directions correspond to the unit cell vectors, but to reduce confusion we use the labels ($x,y,z$). We can approximate the derivative by performing a finite displacement $\Delta u_\alpha(jl)$ and simply taking the numerical derivative


\begin{equation}
\fc _{\alpha \beta} \approx - \frac{F_\beta (\jp \lp; \Delta u_\alpha (jl)) - F_\beta (\jp\lp)}{\Delta u_\alpha (jl)} \label{eq:finite}
\end{equation}


\noindent where $F_\beta (\jp\lp;\Delta u_\alpha (jl))$ is the force on atom $(\jp \lp)$ in the direction $\beta$ after performing the displacement $\Delta u_\alpha (jl)$. At equilibrium we assume $F_\beta (\jp \lp) = 0$ and we only need to calculate the forces due to a finite displacement. In ab-initio methods, there is no reason to assume a particular shape of the potential energy landscape and we can only expect the harmonic approximation to be valid for small finite displacements. For this reason, displacements have to be chosen large enough so that we are not subject to numerical noise and small enough to avoid anharmonic contributions. In \texttt{Phonopy} the default value is \SI{0.01}{\angstrom}, while \texttt{PHONON} uses \SI{0.03}{\angstrom}.

As mentioned in section \ref{sec:phononpractical}, we need a large number of force constants to construct the dynamical matrix, even when dealing with small systems. For this reason, a numerical fitting of forces and displacements, known as the Parlinski-Li-Kawazoe method \cite{Parlinski1997}, is used to find force constants. We notice that equation \eqref{eq:finite} can be written as a matrix equation for one pair of atoms $(jl)$ and $(\jp \lp)$:

\[ \bm{F}(\jp \lp) = -\bm{U}(jl) \bm{P} (jl;\jp \lp) \, , \]

\noindent where

\[ \bm{F}(\jp \lp) = (F_x \quad F_y \quad F_z) \, , \]

\[ \bm{U}(jl) = \left( \Delta u_x(jl) \quad \Delta u_y(jl) \quad \Delta u_z(jl) \right)  \]

\noindent and

\[ 
\bm{P} (jl;\jp \lp) = 	
\begin{pmatrix}
\Phi_{xx} & \Phi_{xy} & \Phi_{xz} \\
\Phi_{yx} & \Phi_{yy} & \Phi_{yz} \\
\Phi_{zx} & \Phi_{zy} & \Phi_{zz} 
\end{pmatrix}
\]

\noindent If we perform $m$ finite displacements, we get $m$ simultaneous equations for each pair of atoms:

\begin{equation*}
\begin{pmatrix} \bm{F}_1(\jp\lp) \\ \bm{F}_2(\jp\lp) \\ \vdots \\ \bm{F}_m(\jp\lp) \end{pmatrix} =
- \begin{pmatrix} \bm{U}_1(jl) \\ \bm{U}_2(jl) \\ \vdots \\ \bm{U}_m(jl) \end{pmatrix} \bm{P}(jl;\jp \lp) \, .
\end{equation*}

\noindent which can be solved by a Moore-Penrose pseudo-inverse matrix (in Numpy: \texttt{numpy.linalg.pinv}) given a sufficient number of displacements. Since $\bm{U}$ only depends on $(jl)$, we can build up the full force constant matrix by iterating this procedure over $(\jp \lp)$. The minimum number of displacements is equal to the number of non-equivalent atoms in the crystal primitive unit cell multiplied by a number of independent x,y,z coordinates in the site symmetry of a given atom \cite{Parlinski1997}. For this reason, software such as \texttt{Phonopy} and \texttt{PHONON} determines the primitive unit cell, the supercell expansion matrix and all the symmetry operations before generating displacements.

\section{Molecular Dynamics}
Temperature fluctuations in a closed system:
%
\[ \frac{\Delta T}{T} = \sqrt{\frac{2}{3N}} \quad \Rightarrow \quad \Delta T(T=\SI{300}{\kelvin}) = \SI{23.1}{\kelvin} \]
%
Nose-Hover algorithm:
%
\begin{align*}
v_i^\text{scaled} &= \frac{\bm{p}_i}{m_i} \\
\text{d}\bm{p}_i &= (\bm{F}_i - \zeta \bm{p}_i) \text{d}t \\
\text{d}\zeta &= \frac{1}{Q} \left( \sum_{i=1}^N \frac{|\bm{p_i}|^2}{m_i} - (3N+1)k_\text{B}T\right) \text{d}t
\end{align*}

\section{Comparing Simulation and Experiment}

\subsection{VACF and gDOS}
It turns out [dove] that the phonon density of states can be found from the power spectrum of the mass-weighted velocity autocorrelation function (VACF). The VACF is, as the name implies, an expression of the correlation between velocites at different. In liquids, the VACF will rapidly decay to zero and in solids the VACF will fluctuate due to coherent vibrations around equilibrium positions. The VACF for a single atomic species $\alpha$ is defined as

\[ C_\alpha(t) = \frac{1}{3} \langle \bm{v}_\alpha(t_0) \cdot \bm{v}_\alpha(t) \rangle \]

\noindent By invoking ergodicity, the ensemble average can be replaced with a time average with respect to $t_0$.

\[ C_\alpha(t) = \frac{1}{3} \frac{1}{T-t} \sum_{t_0=0}^{T-t} \bm{v}_\alpha(t_0) \cdot \bm{v}_\alpha(t_0 + t) = \frac{1}{3} \frac{1}{T-t} \sum_{t_0=0}^{T-t} \sum_{\beta=x,y,z} v_{\alpha\beta}(t_0) v_{\alpha\beta}(t_0 + t) \]

\noindent where $T$ is the total simulation time and $v_{\alpha\beta}$ is the velocity of atom $\alpha$ in direction $\beta$. The total VACF is defined as

\[ C(t) = \sum_\alpha m_\alpha C_\alpha(t) \]

\noindent where $m_\alpha$ is a weight depending on the atomic species and the terms of the sum are the \emph{partial} VACFs. To get the mass-weighted VACF $m_\alpha$ is equal to the atomic weight of species $\alpha$ divided by the average weight of atomic species. The density of states is obtained as the power spectrum of the mass-weighted velocity autocorrelation function:
%
\[ \text{DOS}(\omega) = \sum_\alpha w_\alpha \frac{1}{2\pi} \int^{+\infty}_{-\infty} \mathrm{d}t \exp[-i\omega t] m_\alpha C_\alpha(t) \, , \]
%
where the terms of the sum, once again, determines the \emph{partial} density of states for atomic species $\alpha$. To get the neutron-weighted DOS, we set $w_\alpha$ equal to the coherent neutron cross-section of species $\alpha$. In practice both the VACF and DOS is found numerically by FFT methods.\todo{some details on this since I actually did it myself}

\subsection{Radial distribution function}

\begin{figure}
	\centering
	\includegraphics[width=0.4\textwidth]{fig/temp/gr.png}
	\label{fig:rdf}
	\caption[RDF visualization]{Visualization of the radial distribution function}
\end{figure}

The radial distribution function is defined as
%
\[ g(r) = \frac{\rho(r)}{\rho} \, , \]
%
where $\rho(r)$ is the particle number density at a distance $r$ from an arbitrary atomic origin and $\rho = \frac{N}{V}$ is the number density of the unit cell. A visual representation of this quantity is shown in Figure \ref{fig:rdf} Formally this can be found from atomic positions
%
\[ g(r) = \frac{1}{N\rho} \left\langle \sum_i^N \sum_{i \neq j} \delta(r - r_{ij})  \right\rangle \]
%
where $i$ and $j$ are particle indices and $r_{ij}$ is the distance between particle $i$ and $j$. $N$ is the total number of particles and the brackets denote an ensemble average. Due to the delta function, this expression is not particularly useful when analysing MD trajectories of discrete particle positions. To overcome this, we define $n(r,\mathrm{d}r)$ as a function that counts the number of particles at a distance $r$ within a spherical shell of thickness $\mathrm{d}r$.
%
\begin{equation}\label{eq:g_of_r}
g(r) = \frac{2 \left\langle n(r,\mathrm{d}r) \right\rangle}{N \rho V_\text{s}(r,\mathrm{d}r)} \, ,
\end{equation}
%
where $V_\text{s}(r,\mathrm{d}r) \approx 4 \pi r^2 \mathrm{d}r$ is the thickness of the spherical shell. Due to the ergodic hypothesis, the ensemble average is replaced with a time average, so that 
%
\[ \left\langle n(r,\mathrm{d}r) \right\rangle = \frac{1}{M} \sum_{k=1}^M n_k(r,\mathrm{d}r) \, \]
%
where $M$ is the number of time steps. In practice $n_k$ is evaluated by creating a list of all particle-particle distances in the frame $k$ and generating a histogram with bin size $\mathrm{d}r$. The factor of two in equation \eqref{eq:g_of_r} comes from the fact that $n_k$ only counts each pair once.

The $g(r)$ we just defined treats all particle pairs on an equal footing. In order to compare simulations with neutron scattering data it is necessary to weigh distinct particle pairs by the product of their neutron cross sections. First, we define the partial radial distribution function $g_{\alpha\beta}(r)$ as the probability of finding a particle with label $\beta$ at a distance $r$ away from a particle with label $\alpha$ plus the probability of finding a particle with label $\alpha$ at a distance $r$ away from a particle with label $\beta$.
%
\begin{align*}
g_{\alpha\beta}(r) &= \frac{ \left\langle n_{\alpha\beta}(r,\mathrm{d}r) \right\rangle}{N_\alpha  \rho_\beta V_\text{s}(r,\mathrm{d}r)} + \frac{ \left\langle n_{\alpha\beta}(r,\mathrm{d}r) \right\rangle}{N_\beta \rho_\alpha V_\text{s}(r,\mathrm{d}r)} \\
&= \frac{2V}{N_\alpha N_\beta} \frac{ \left\langle n_{\alpha\beta}(r,\mathrm{d}r) \right\rangle}{V_\text{s}(r,\mathrm{d}r)} \, ,
\end{align*}
%
where $N_i$ is the number and $\rho_i$ is the density of particle species $i$. The reason to define it in `both directions' is that $n_{\alpha\beta}$ is symmetric to exchange of particles. From the partial pair distribution functions, $g(r)$ can be trivially computed and optionally weighted by neutron cross sections:
%
\[ G(r) = \frac{1}{\overline{b}^2} \sum_{\alpha=1,\beta\geq\alpha} c_\alpha c_\beta \overline{b}_\alpha \overline{b}_\beta (g_{\alpha\beta}(r) - 1) \, , \]
%
where $c_i = \frac{N_i}{N}$ is the number concentration and $\bar{b_i}$ is the coherent neutron cross section of species $i$ and 
%
\[ \overline{b}^2 = \left(\sum_\alpha \overline{b}_\alpha c_\alpha \right)^2 \, . \]
%
In the PDF community there is a large number of conventions regarding how to normalize and represent pair-correlation functions in theory and experiment (see review by keen et al.). The $G(r)$ presented here is typically called the \emph{Total Pair-Distribution Function}, but your mileage may vary. For the calculations and experiments presented in this thesis we always refer to $g_{\alpha\beta}(r)$ and $G(r)$ as defined here.

\subsection{Atomic distance histograms}
If our MD simulations fulfils the ergodic hypothesis, it is useful to look at various distributions that can be extracted from the simulation. While a lot of this information is contained in the radial distribution function, the system studied in this thesis is not isotropic. In fact, the 2-dimensional nature of the cuprates appears to be essential for the electronic properties. By generating histograms for certain atoms, we can ask a few pertinent questions such as:
%
\begin{itemize}
	\item What is the distribution of Cu-O$_\text{eq}$ distance?
	\item What is the distribution of Cu-O$_\text{ap}$ distance?
	\item What is the distribution of octahedral tilts ($Q_1$, $Q_2$)?
	\item What is the nature of O$_\text{int}$ diffusion?
\end{itemize}
%
All of these questions are well-defined in the context of molecular dynamics trajectories, but it can be tedious for large systems to label all the relevant atoms. To overcome this, we generate pairs of atomic species based on certain conditions. For example, if we want to find pairs of Cu and O$_\text{eq}$, we loop over all Cu-O pairs and only list the pairs where the distance vector is less than $\bm{r} = (2.1, 2.1, 1)\,\si{\angstrom}$. After building the pair-lists it is trivial to generate histograms of certain distances.

Similarly, we can build the CuO$_6$ octahedra by applying the same idea to both equatorial and apical oxygen atoms. We can identify the 6 corners of the octahedron simply by checking the signs of the 6 distance vectors (e.g. the `top' apical oxygen will have a positive $z$ component). $Q_1$ and $Q_2$ can then be computed and we can generate histograms of the octahedral tilts.

Finally, we can also use these pairs to generate symmetry operations in a fairly simple way. Since we know that the octahedra have alternating tilt patterns, we can check the tilt pattern at frame 1 and generate a list of the 4 different combinations of $Q_1$ and $Q_2$ ((+,+),(+,-),(-,+),(-,-)). Applying these symmetry operations to our calculations then lets us obtain a histogram of the symmetry-adapted octahedral tilts.

\subsection{Types of measurements}
In this thesis 3 types of neutron measurements has been performed:

\begin{enumerate}
	\item Direct measurements of phonon bands with TAS spectroscopy
	\item Phonon density of states on powders with time-of-flight methods.
	\item PDF measurements of powders
\end{enumerate}

\noindent All of these can be compared directly with simulations in various ways, but there are some subtleties on how to perform this comparison correctly. In this section we thus treat them one at a time.

\subsubsection{Phonon bands with TAS}
The master equation for phonon scattering can be found in [squires] and directly relates the measured differential cross-section to the phonon band structure:
%
\begin{equation}
	S(\bm{Q},\nu,\omega) = \frac{k_\text{f}}{k_\text{i}} \frac{N}{\hbar} \sum_{\bm{q}} | F(\bm{Q},\bm{q},\nu) |^2 ( n_{\bm{q}\nu} + 1) \delta (\omega - \omega_{\bm{q}\nu}) \delta(\bm{Q} - \bm{q} - \bm{G})
	\label{eq:one_phonon_sqw}
\end{equation}
%
with
%
\[
	F(\bm{Q}, \bm{q}, \nu) = \sum_j \sqrt{\frac{\hbar}{2 m_j \omega_{\bm{q}\nu}}} \bar{b}_j \exp \left( -\frac{1}{2} \langle | \bm{Q} \cdot \bm{u}(j0) |^2 \rangle \right) \exp [ -i(\bm{Q} - \bm{q}) \cdot \bm{r}(j0) ] \bm{Q} \cdot \bm{e}_j(\bm{q},\nu)
\]
%
where $\bm{Q}$ and $\omega$ are the wave vector and energy of our measurement. $\nu$ is a phonon band and $\bm{q}$ is a wave vector. $\omega_{\bm{q}\nu}$ is the energy of phonon band $\nu$ at wave vector $\bm{q}$ and $n_{\bm{q}\nu}$ is the Bose factor at this energy. Atomic species are designated with index $j$, their mass is $m_j$ and their coherent neutron cross-section is $\bar{b}_j$. $\bm{u}(j0)$ and $\bm{r}(j0)$ is the displacement and position of atom $j$ in unit cell 0, respectively. Finally, $G$ is any reciprocal lattice vector and $\bm{e}_j(\bm{q},\nu)$ is the phonon eigenvector of atom $j$, band $\nu$ and wave vector $\bm{q}$.

Note that the equations have been rewritten slightly when compared to Squires (similar to what is presented on the Phonopy website), such that $S(\bm{Q},\nu,\omega)$ is defined separately for each phonon band. In addition, the sum in the phonon structure factor runs over atomic indices in unit cell 0, consistent with the definitions made earlier in this chapter.

By close inspection of equation \eqref{eq:one_phonon_sqw}, we realise that the information obtained by the calculation of phonon band structures provides us with all the information necessary to construct $S(\bm{Q},\nu,\omega)$. Since normalization on an absolute scale is usually not possible when performing a TAS experiment, we can set $N k_\text{f} / k_\text{i} = 1$ to simplify.

The $\delta$-functions in equation \eqref{eq:one_phonon_sqw} tells us that a neutron measurement will only have intensity if we measure at values of $\bm{Q}$ and $\omega$ that correspond to a point of the dispersion of band $\nu$. We thus reduce our calculations to sampling $S(\bm{Q},\nu,\omega)$ at ($\bm{q},\omega_{\bm{q}\nu}$). The only computationally heavy part then becomes the Debye-Waller factor
%
\[ W = \frac{1}{2} \left\langle | \bm{Q} \cdot \bm{u}(j0) |^2 \right\rangle \]
%
which has to be sampled at some finite grid in unit cell 0 in order to get a reasonable estimate of the ensemble average. In many cases we compare measurements to a phonon dispersion in the first BZ where $W$ varies only slightly, so if we want to sample a large number of $\bm{Q}$-points it can be advantageous to simply omit the Debye-Waller factor. As of this writing, this is not possible in Phonopy directly, so we have to live with someone heavy computations for now.

With these equations in mind, we can now plot the neutron-weighted phonon bands and compare them with TAS-measurements. We can represent the neutron weighted bands either by colouring the band-structure lines according to intensity or by giving the dispersion curves a finite Gaussian width to replicate a finite instrument resolution and/or linewidth broadening. Figure \ref{fig:bands_sqw_color_line} shows examples of the two kinds of representations. By adding obtained neutron data to these plots, we can then directly compare theory and experiment. We note here that a comparison with MD simulations is not possible since these simulations give no information about the discrete phonon bands.

\begin{figure}
	\centering
	\missingfigure{left: bands lineplot, right: bands colorplot}
	\caption[Neutron weighted bands example]{Neutron weighted bands example}
	\label{fig:bands_sqw_color_line}
\end{figure}

\subsubsection{Phonon Density of States}
The phonon density of states (DOS) is a simple projection of the phonon bands onto the energy axis. While this obviously reduces the amount of information due to the reduction of dimensionality, the phonon DOS is a useful object for a couple of reasons:

\begin{enumerate}
	\item Often only powders are available for experiments and resolving bands can be difficult due to the rotational averaging.
	\item Many neutron scattering instruments are specifically designed for DOS measurements.
	\item DOS can be obtained from molecular dynamics as well as band structure calculations.
\end{enumerate}

\noindent The last point is particularly important in the case where we are working with defect structures such as oxygen interstitials. Here, the symmetry is usually severely broken (often to P1 symmetry) and a phonon calculation would be prohibitively expensive. As shown in section XX\todo{ref}, the phonon DOS can be obtained, in the harmonic approximation, rather simply from a MD trajectory.

To obtain the DOS from a band-structure calculation, we perform an integration over a commensurate grid in the 1st BZ and project the result onto the energy axis for each atomic species $j$ in the following way:
%
\begin{align*}
	g^j (\omega) &= \sum_{\bm{\hat{n}}=\{x,y,z\}} \frac{1}{N} \sum_{\bm{q},\nu} \delta(\omega - \omega_{\bm{q}\nu}) \left| \bm{\hat{n}} \cdot \bm{e}_j(\bm{q},\nu) \right| ^2 \\
	g(\omega) &= \sum_j g^j(\omega)
\end{align*}
%
where $\hat{n}$ is the unit vector in the three cartesian directions. Comparing this equation with equation \eqref{eq:one_phonon_sqw} we notice that we only need to weigh by mass and the neutron cross-section in order to go from the true density of states $g(\omega)$ to the neutron $S(\omega)$\todo{look over these details once more}. If we performed a TAS experiment on a single crystal with perfect resolution at every relevant $(\bm{q},\omega_{\bm{q}\nu})$ point, these definitions would be correct and we could perform the integration on our massive 4-dimensional dataset. However, due to the nature of typical DOS measurements, we need to invoke the so-called \emph{incoherent approximation} and use the incoherent 1-phonon partial differential cross-section when treating the data. In practice this means using the total neutron cross sections ($\sigma_\text{tot}$) with the incoherent 1-phonon $S(\bm{q},\omega)$ \todo{expand on this, I still find it confusing}.

\subsubsection{Pair-density function measurements}
When performing neutron PDF measurements, it is possible to extract the normalized total neutron PDF (construction of partials require several measurements, see []\todo{ref}). As such, we should be able to compare directly with the PDF as extracted from molecular dynamics as shown in section XX\todo{ref}. PDF from phonon band structure calculations are in principle possible since we can extract thermal displacements, but it is currently not a feature in Phonopy\todo{maybe do this? Could be interesting.}, so this analysis has not been performed here. 
