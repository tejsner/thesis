\chapter{Molecular Dynamics}\label{ch:md}

Using the lessons learned in the previous chapter and after a reasonably successful validation of simulations (see section XX), we can use the computational parameters at a lower precision and perform Molecular Dynamics simulations in VASP. The purpose of this chapter is to outline the chosen defect structures, discuss the computational details and present how defects (interstitial or otherwise) modifies the structure and dynamics of the parent structure.


\section{Molecular Dynamics}
\[ T = \frac{1}{3 k_\text{B} (N_\text{ions}-1)} \sum_n M_n |\bm{v}_n|^2 \]
\todo[inline]{in progress... Data from LTO and LTT exists...}

\begin{figure}
	\centering
	\includegraphics[width=\textwidth]{fig/md/stitch.pdf}
	\caption[stitched md runs]{sticthed md runs and comparison}
	\label{fig:stitch}
\end{figure}

\begin{figure}
	\centering
	\includegraphics[width=\textwidth]{fig/md/lto_ltt_ltoo_comparison.pdf}
	\caption[MD: DOS and PDF]{MD: DOS and PDF}
	\label{fig:dos_pdf}
\end{figure}

\begin{figure}[]
	\centering
	\includegraphics[width=\textwidth]{fig/md/lto_defect_comparison.png}
	\caption[LTO MD DOS: Defect comparision]{LTO MD DOS: Defect comparision}
	\label{fig:lto_md_defect_comparison}
\end{figure}

\begin{figure}
	\centering
	\includegraphics[width=\textwidth]{fig/md/dist_hist.pdf}
	\caption[MD: distance histograms]{MD: distance histograms}
	\label{fig:md_distances}
\end{figure}

\begin{figure}
	\centering
	\includegraphics[width=\textwidth]{fig/md/prm_q1q2.pdf}
	\caption[MD Q1 Q2]{MD Q1 Q2}
	\label{fig:md_q1_q2}
\end{figure}

\begin{figure}
	\centering
	\includegraphics[width=\textwidth]{fig/md/Q1_Q2_all.png}
	\caption[MD Q1 Q2 All sims]{MD Q1 Q2 All sims}
	\label{fig:md_q1_q2_all}
\end{figure}

\begin{table}
	\centering
	\begin{tabular}{lllllll}
		\toprule
			name &   d(Oeq) & sigma(Oeq) &  R2(Oeq) &   d(Oap) & sigma(Oap) &  R2(Oap) \\
		\midrule
			 LTO &  1.89837 &  0.0454907 &  0.997277 &  2.42126 &   0.112764 &  0.998931 \\
		   LTO+O &  1.90227 &  0.0464374 &  0.997236 &  2.40559 &   0.127141 &  0.960830 \\
		  LTO+Sr &  1.89838 &  0.0507032 &  0.994707 &  2.43771 &   0.146267 &  0.999500 \\
			LTOm &  1.90897 &  0.0513818 &  0.995099 &  2.46200 &   0.140257 &  0.999679 \\
		  LTOm+O &  1.90826 &  0.0532324 &  0.993406 &  2.43961 &   0.155845 &  0.999179 \\
		 LTOm+Sr &  1.90508 &  0.0522777 &  0.994720 &  2.45116 &   0.150613 &  0.999230 \\
			 LTT &  1.90732 &  0.0460712 &  0.997086 &  2.40577 &   0.108611 &  0.999185 \\
		\bottomrule
		\end{tabular}
	\caption{MD: Cu-O Distances}
	\label{tab:md_cu_o_distances}
\end{table}

\begin{figure}
	\centering
	\includegraphics[width=\textwidth]{fig/md/diffusion1.pdf}
	\caption[MD Oint Diffusion Oi]{MD Oint Diffusion Oi}
	\label{fig:md_diffusion1}
\end{figure}

\begin{figure}
	\centering
	\includegraphics[width=\textwidth]{fig/md/diffusion2.pdf}
	\caption[MD Oint Diffusion Oap 1]{MD Oint Diffusion Oap 1}
	\label{fig:md_diffusion2}
\end{figure}

\begin{figure}
	\centering
	\includegraphics[width=\textwidth]{fig/md/diffusion3.pdf}
	\caption[MD Oint Diffusion Oap 2]{MD Oint Diffusion Oap 2}
	\label{fig:md_diffusion3}
\end{figure}


\begin{figure}
	\centering
	\includegraphics[width=\textwidth]{fig/md/md_phonopy_comparison.pdf}
	\caption[MD Phonopy Comparison]{MD Phonopy Comparison}
	\label{fig:md_phonopy_comparison}
\end{figure}

\section{LCO+O Phonons}
Since the introduction of interstitials breaks the local crystal symmetry, we cannot perform phonon calculations using the direct method without making an excessive amount of displacement calculations. For this reason, we are forces to use ab-initio molecular dynamics.

Before thinking about molecular dynamics, we need to place the interstitial. Figure \ref{fig:oint_location} shows possible interstitial positions for the LTO and LTT phase. We ignore the HTT phase for molecular dynamics since the symmetry is broken and, in some sense, LTT and HTT are identical (volume and a/c ratio are very similar).

\begin{figure}
    \centering
    \includegraphics[width=\textwidth]{fig/md/oint.png}
    \caption[Illustration of interstitial positions]{Illustration of interstitial oxygen in-plane ($a$-$b$) location with respect to the apical oxygen displacements in the rock-salt layer. Open squares represent apical oxygens `hanging down' while closed squares represent apical oxygens `sticking up'. Interstitial oxygen are red circles. Adapted from \cite{Tranquada1995}.}
    \label{fig:oint_location}
\end{figure}

To see how the symmetry is broken, Table \ref{tab:oint_locations} shows the resulting space groups as a function of interstitial $z$-component and various defect structures in a $2 \times 2 \times 1$ supercell based on the orthorhombic crystallographic cell. 

\begin{table}[b]
    \centering
    \begin{tabular}{@{}lllll@{}}
\toprule
Phase                   & Space Group & $\text{O}_\text{i}^x$ & $\text{O}_\text{i}^y$ & $\text{O}_\text{i}^z$ \\ \midrule
HTT                     & I4/mmm (139)      &         &         &         \\
HTT + O$_\text{i}$      & P-42m  (111)    & 0.125   & 0.125   & 0.25    \\
HTT + O$_\text{i}$      & Cmm2 (35)       & 0.125   & 0.125   & 0.24    \\
LTO                     & Bmab (64)       &         &         &         \\
LTO + O$_\text{i}$      & P2 (3)    & 0.125   & 0.125   & 0.25    \\
LTO + O$_\text{i}$      & P2 (3)         & 0.125   & 0.125   & 0.24    \\
LTO$_\text{defect}$     & Pmna (53)       &         &         &         \\
LTO$_\text{defect}$ + O$_\text{i}$ & P2 (3)         & 0.875   & 0.375   & 0.25    \\
LTO$_\text{defect}$ + O$_\text{i}$ & P2 (3)         & 0.875   & 0.375   & 0.24    \\
LTT                     & P4$_2$/ncm (138)  &         &         &         \\
LTT + O$_\text{i}$      & P-4 (81)        & 0.375   & 0.125   & 0.25    \\
LTT + O$_\text{i}$      & P2 (3)         & 0.375   & 0.125   & 0.24    \\
LTT$_\text{defect}$               & Pmma (51)       &         &         &         \\
LTT$_\text{defect}$ + O$_\text{i}$ & Cmm2 (35)       & 0.875   & 0.375   & 0.25    \\
LTT$_\text{defect}$ + O$_\text{i}$ & Cmm2 (35)       & 0.875   & 0.375   & 0.24    \\ \bottomrule
\end{tabular}

    \caption[Oxygen interstitial phases]{Space group symmetry due to the introduction of an interstitial oxygen in various structures all described in a $2 \times 2 \times 1$ supercell of the Bmab (conventional) coordinate system. HTT, LTO and LTT are the usual phases as described in litterature \cite{Hucker2012}. The structures labeled defect is (A) in the LTO case: A stacking fault where the middle layer has its tilts reversed and (B) in the LTT case: A line along [110] with reversed tilts. Both are described in \cite{Tranquada1994} and are designed in order to `make room' for the interstitial oxygen (see Figure \ref{fig:oint_location}).}
    \label{tab:oint_locations}
\end{table}

We performed high-precision geometry optimizations (with symmetry turned off) on these structures (including a test of HTT) resulting in the table of energies as shown in Table \ref{tab:oint_en}. While the defect structures intuitively would make more room for oxygen interstitials, the energy cost of forming the defect is higher in the, relatively small, supercell that we use. It is worth noting, however, that they move to very similar energies during the optimization\todo{not so surprising, since they probably move towards the same structure. I need to check this more closely}.


\begin{table}[b]
    \centering
    \begin{tabular}{@{}lll@{}}
    \toprule
     & $E_0$ [eV] & $E_1$ [eV]            \\ \midrule
    HTT                     & -827.27669             & -829.76250 \\
    LTO                     & -828.29890             & -830.39658 \\
    LTO\_sfault              & -823.09516             & -830.03588 \\
    LTT                     & -828.04663             & -830.08248 \\
    LTT\_defect              & -826.03173             & -829.94243 \\ \bottomrule
    \end{tabular}
    \caption[Oxygen interstital phases: Energy]{Oxygen interstital phases: Energy. $E_0$ corresponds to the energy after inserting the interstitial oxygen, but before geometry optimization. $E_1$ is the total energy after optimization. Geomtry optimization performed on ionic positions only. Conclusion: It costs more energy to create the defect structure than you gain by making room for the interstitial.}
    \label{tab:oint_en}
\end{table}