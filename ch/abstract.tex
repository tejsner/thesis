\chapter*{Abstract}
Understanding the mechanisms of high-temperature superconductivity in the cuprates remains one of the great unsolved questions of condensed matter physics. Hole-doping the Mott-insulator La$_2$CuO$_4$ with various chemical species creates a superconductor with a transition temperature $T_\text{c}$ that depends heavily on the amount and type of doped species. While the amount of hole-doping typically dictates the phase diagram, there are certain situations where superconductivity can be suppressed or enhanced due to structural modifications caused by the dopant species. In this thesis, I investigate these structural modifications and, in particular, related dynamics in order to better understand any possible relationship to superconductivity.

A particular interesting sample in this context is La$_{2-x}$Sr$_x$CuO$_{4+\delta}$ (LSCO+O), where doping can be performed with two distinct chemical species, Sr and O. Doping with Sr is `quenched', meaning that Sr has a fixed, random distribution on La sites. On the other hand, doping with O is performed after crystal growth using electrochemical methods. Oxygen dopants are `annealed' in the sense that they sit on interstitial sites and are mobile at room temperature. In addition, doping with O creates a superconductor with slightly better $T_\text{c}$ that is equipped with a number of unique phenomena such as electronic phase separation and complex superstructures. In this thesis, various aspects of phonon dynamics in LSCO+O are investigated through a combination of theoretical and experimental methods. Density functional theory and molecular dynamics is used to numerically estimate the phonon band structure and density of states. Inelastic neutron scattering is used to probe phonon dynamics and the two methods are compared.

Despite the fact that Density Functional Theory is known to struggle with the electronic structure of the cuprates, excellent agreement between phonon dynamical simulations and experiments are found. An analysis of this relationship reveals unique dynamics of octahedral tilt patterns due to interstitial oxygen. We speculate that these tilt patterns may assist in relieving the inherent frustration between magnetism and superconductivity in LSCO+O. Finally, anomalous features in the Cu-O bond stretching phonon of two different LSCO+O samples is observed with neutron scattering. This `phonon anomaly' is suspected to be a signature of charge density fluctuations that might be important for superconductivity. While previously observed in Sr-doped La$_{2-x}$Sr$_x$CuO$_4$, this result rules out a phonon anomaly caused by structural mechanism related to the specific dopant type.